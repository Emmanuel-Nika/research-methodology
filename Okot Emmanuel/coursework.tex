\documentclass[12pt]{article}
\usepackage{zed-csp}
\usepackage[top=2.5cm, bottom=2.5cm, left=3cm, right=3cm]{geometry}
\usepackage{graphicx}
\begin{document}

\begin{Huge}
\begin{center}
\begin{normalsize}
\textbf{MAKERERE \includegraphics[scale=0.5]{logo} UNIVERSITY }\\


\textbf{FACULTY OF COMPUTING AND INFORMATICS TECHNOLOGY} \\
\textbf{SCHOOL OF COMPUTING AND INFORMATICS TECHNOLOGY} \\
\textbf{DEPARTMENT OF COMPUTER SCIENCE} \\
\textbf{BACHELOR OF SCIENCE IN COMPUTER SCIENCE} \\
\textbf{YEAR 2} \\
\textbf{BIT 2207 RESEARCH METHODOLOGY} \\
\textbf{Course Work: Assignment 1}\\
\end{normalsize}
\end{center}
\end{Huge}

\begin{center}
\begin{tabular}{|l|l|l|c|}
\hline NAME  & REG NO & STD NO \\\hline
OKOT EMMANUELL& 16/U/10916/PS & 216015844 \\\hline
\end{tabular}

\end{center}

\newpage

\begin{center}
\textbf{INCREASING UNEMPLOYMENT OF COMPUTER SCIENCE GRADUATES IN THE WORLD}\\
\paragraph{•}
Prepared by: OKOT EMMANUEL\\
\paragraph{•}
Lecturer: ERNEST MWEBAZE \\
\paragraph{•}
7th January 2018

\end{center}

\section{Abstract}
\paragraph{•}
The aim of this study was to explore how disconcerting facts about the employment prospects of highly educated computer science graduates in the world. This research is intended to see how graduates are faring two years after completing various courses. At the time, employment statistics show incredibly low levels of unemployment for computer science graduates in official statistics. However, unemployment rates amongst those who graduated for example in Ireland in 2011, in computing courses are higher than almost any other degree or diploma course.
\paragraph{•}
Data was collected from various websites and at the end of the research work, some conclusions were made. 

\section{Introduction}
\paragraph{•}
During the late 1990s undergraduate entrants studying full-time on computer science courses at English higher education institutions increased rapidly peaking at around 29,000 in 2002-03. Over the subsequent four years entrant numbers quickly declined to around 17,500 by 2006-07 before returning to more modest rates of growth. 
\paragraph{•}
In each of the last six years, more students have begun computer science courses than physics, chemistry and maths combined. So, there was obvious concern when it was noted that of undergraduates who qualify across all higher education subjects, computer science has consistently had the highest rate of unemployed graduates. 
\paragraph{•}
There are concerns about the proportion of undergraduate computer science students who progress into low-paid or non-graduate level employment. There are concerns over the reliance our computer science departments have on international recruitment to fill their labs and postgraduate courses. And there are concerns from industry about the skills, agility and work-readiness of those computer science graduates flowing into the workforce.
Many will know already that computer science has a distinctive profile of students. Notably it has more men, students from black and minority ethnic backgrounds, and students with lower previous levels of attainment.
\paragraph{•}
The statistics show that unemployment among black and minority ethnic graduates from full-time, first degrees is six percentage points higher than among white graduates. But interestingly this difference is smaller than it is for graduates from comparable subjects like electronic and electrical engineering or mathematical sciences. 
\paragraph{•}
The “disadvantaged” students are usually those who come from neighborhoods with low levels of participation in higher education. The percentage level of unemployment among full-time graduates from computer science courses who come from disadvantaged backgrounds is only one percent point higher than for the majority of STEM subjects. 
\paragraph{•}
And from a different angle – gender – the differences disappear. The unemployment rate for men and women graduating from computer science is the same.
\paragraph{•}
What about those students who are employed? Does this show us anything distinctive about the profile of computer science graduates? 
\paragraph{•}
Around 65 per cent of computer science qualifiers who enter full-time paid employment within six months of leaving HE tell us that they earn less than £25,000. This compares with 55 per cent of equivalent qualifiers from electronic and electrical engineering courses, and with 60 per cent of those qualifying from mathematical sciences. 
\paragraph{•}
When we look at the sectors of industry in which graduates are employed, a high proportion (38 per cent) work in the information and communication industry. The rest are dispersed across a wide range of industries. 
\paragraph{•}
Among those who gain employment in a micro, small- or medium-sized business, around 86 per cent occupy a professional or managerial role within that business. For those in the largest firms, 68 per cent were in these types of roles. This means that a computer science graduate gaining employment in a larger firm is more likely to work in an administrative, trade, service, sales, plant operative or elementary role than their counterpart who is working in a smaller firm. 
\paragraph{•}
The story is similar for those qualifying from electronic and electrical engineering, though the differences between smaller and larger firms are less.
\paragraph{•}
But for maths graduates those employed in large and medium firms were the ones most likely to be in the professional and managerial roles, with their counterparts in smaller firms being least likely to be in these types of roles.

\section{Methods}
\paragraph{•}
The research was carried over several websites. Not only was this research carried out from those websites that address specifically issues concerning with unemployment among computer science graduates, but also from the general causes of unemployment amongst graduates from higher education.
\section{Conclusion}



\end{document}